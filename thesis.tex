%% ----------------------------------------------------------------
%% Thesis.tex -- MAIN FILE (the one that you compile with LaTeX)
%% ---------------------------------------------------------------- 

%% -----  PRINT VERSION ----------------------------------------------------------
\documentclass[a4paper, 11pt, twoside, openright]{UPMthesis}
\hypersetup{urlcolor=black, colorlinks=true, linkcolor={black},citecolor={black}}

%~ %% -----  DIGITAL VERSION --------------------------------------------------------
%~ \documentclass[a4paper, 11pt, oneside]{UPMthesis}
%~ \hypersetup{urlcolor=black, colorlinks=true, linkcolor={blue!50!black},citecolor={blue!50!black}}
%~ %% -------------------------------------------------------------------------------

% Biblatex
\usepackage[doi=false,isbn=false,url=false,maxcitenames=2,mincitenames=1, style=numeric-comp,defernumbers=true, 
            maxbibnames=4,minbibnames=3,firstinits=true,sorting=none]{biblatex}  % Use new bibliography styles
\AtEveryBibitem{\clearlist{language}} % clears language
\AtEveryBibitem{\clearlist{address}} % clears address
\addbibresource{references.bib}

% Acronyms
\usepackage[acronym,toc,nonumberlist,shortcuts]{glossaries}
\glsdisablehyper
\makeglossaries

\newacronym{3fhd}{3F--HD}{3-Finger Haptic Device}
\newacronym{ac}{AC}{Alternating Current}
\newacronym{amqp}{AMQP}{Advanced Message Queuing Protocol}
\newacronym{anova}{ANOVA}{ANalysis Of VAriance}
\newacronym{api}{API}{Application Programming Interface}
\newacronym{blos}{BLOS}{Beyond Line-Of-Sight}
\newacronym{cad}{CAD}{Computer Aided Design}
\newacronym{car}{CAR}{Centre for Automation and Robotics}
\newacronym{cern}{CERN}{European Organization for Nuclear Research}
\newacronym{csic}{CSIC}{Consejo Superior de Investigaciones Cient\'{i}ficas}
\newacronym{collada}{COLLADA}{COLLAborative Design Activity}
\newacronym{daq}{DAQ}{Data Acquisition}
\newacronym{darpa}{DARPA}{Defense Advanced Research Projects Agency}
\newacronym{dc}{DC}{Direct Current}
\newacronym{dh}{DH}{Denavit and Hartenberg}
\newacronym{drc}{DRC}{DARPA Robotics Challenge}
\newacronym{drcsim}{DRCSim}{DRC Simulator}
\newacronym{ekf}{EKF}{Extended Kalman Filter}
\newacronym[longplural=Degrees of Freedom]{dof}{DoF}{Degree of Freedom}
\newacronym{ffb}{FFB}{Force Feedback}
\newacronym{fk}{FK}{Forward Kinematics}
\newacronym{gnvff}{GN-VFF-RLS}{Gauss Newton Variable Forgetting Factor Recursive Least-Squares}
\newacronym{gsi}{GSI}{GSI Helmholtz Centre for Heavy Ion Research}
\newacronym{gui}{GUI}{Graphical User Interface}
\newacronym{hc}{HC}{Hunt--Crossley}
\newacronym{hdd}{HDD}{Hard Disk Drive}
\newacronym{iec}{IEC}{Integral Error Criterion}
\newacronym{ida}{I--D}{Iterative Decoupling algorithm}
\newacronym{ik}{IK}{Inverse Kinematics}
\newacronym{imu}{IMU}{Inertial Measurement Unit}
\newacronym{iter}{ITER}{International Thermonuclear Experimental Reactor}
\newacronym{jet}{JET}{Joint European Torus}
\newacronym{jt}{JT}{Jacobian Transpose}
\newacronym{kdl}{KDL}{Kinematics and Dynamics Library}
\newacronym{kv}{KV}{Kelvin--Voigt}
\newacronym{kraft}{Kraft}{Kraft Telerobotics Inc}
\newacronym{lan}{LAN}{Local Area Network}
\newacronym{lhc}{LHC}{Large Hadron Collider}
\newacronym{mocap}{MoCap}{Motion Capture}
\newacronym{nasa}{NASA}{National Aeronautics and Space Administration}
\newacronym{ni}{NI}{National Instruments}
\newacronym{nrmse}{NRMSE}{Normalized Root-Mean-Square Error}
\newacronym{ocu}{OCU}{Operator Control Unit}
\newacronym{op}{Op--Amp}{Operational Amplifier}
\newacronym{orocos}{OROCOS}{Open Robot Control Software}
\newacronym{os}{OS}{Operating System}
\newacronym{phantom}{PHANToM}{Personal HAptic iNTerface Mechanism}
\newacronym{ransac}{RANSAC}{RANdom Sampling Consensus}
\newacronym{rfi}{RFI}{Radio Frequency Interference}
\newacronym{rh}{RH}{Remote Handling}
\newacronym{ros}{ROS}{Robot Operating System}
\newacronym{rpy}{RPY}{Roll--Pitch--Yaw}
\newacronym{rls}{RLS}{Recursive Least-Squares}
\newacronym{rt}{RT}{Real--Time}
\newacronym{rtos}{RTOS}{Real--Time Operating System}
\newacronym{rviz}{RVIZ}{ROS Visualization Tool}
\newacronym{smach}{SMACH}{State-Machine-Based Execution and Coordination System}
\newacronym{sprls}{SPRLS}{Self-Perturbing Recursive Least-Squares}
\newacronym{ssd}{SSD}{Solid-State Drive}
\newacronym{svd}{SVD}{Singular Value Decomposition}
\newacronym{temar}{TEMAR}{Remote Manipulation Techniques for Nuclear Fusion Research Centers (In Spanish: \textit{T\'{e}cnicas de Manipulaci\'{o}n Remota para Centros de Investigaci\'{o}n de Fusi\'{o}n Nuclear})}
\newacronym{uav}{UAV}{Unmanned Aerial Vehicle}
\newacronym{udp}{UDP}{User Datagram Protocol}
\newacronym{upm}{UPM}{Universidad Polit\'{e}cnica de Madrid}
\newacronym{urdf}{URDF}{Unified Robot Description Format}
\newacronym{tcp}{TCP}{Transmission Control Protocol}
\newacronym{tear}{TEAR}{TCP Emulation At Receivers}
\newacronym{vff}{VFF}{Variable Forgetting Factor}
\newacronym{vpn}{VPN}{Virtual Private Network}
\newacronym{vrc}{VRC}{Virtual Robotics Challenge}



\newacronym{frvf}{FRVF}{Forbidden Region Virtual Fixture}
\newacronym{gvf}{GVF}{Guidance Virtual Fixture}




\newacronym{sql}{SQL}{Structured Query Language}
\newacronym{dbms}{DBMS}{Database Management System}

\newacronym{lti}{LTI}{Linear Time Invariant}




% The package glossaries defines a boolean flag for every entry. If you call \ac the boolean flag is set to true. That means next time you know that the entry was used.
% To set this flag manual the package provides the command \glsunset.
% The documentation describes the command at page 105 (glossaries-user.pdf).
\glsunset{3fhd}
\glsunset{ac}
\glsunset{anova}
\glsunset{api}
\glsunset{csic}
\glsunset{darpa}
\glsunset{dc}
\glsunset{dof}
\glsunset{ffb}
\glsunset{lan}
\glsunset{nasa}
\glsunset{phantom}
\glsunset{rviz}
\glsunset{tcp}
\glsunset{temar}
\glsunset{udp}
\glsunset{upm}
\glsunset{upm}
\glsunset{vpn}
\glsunset{ni}
\glsunset{hdd}
\glsunset{ssd}


% Extra-packages
\usepackage{afterpage}
\usepackage[chapter]{algorithm}
\usepackage{algorithmic}
\usepackage{csquotes}
\usepackage{fixmath}
\usepackage{pdfpages}
\usepackage{pgfplots}
\usepackage{rotating}
\usepackage{tabu}
\usepackage{tikz}
\usepackage{verbatim}
\usepackage{blindtext}

\graphicspath{{./figures/}}
\DeclareGraphicsExtensions{.pdf,.jpg,.jpeg,.png}

% Hesitation macros
\newcommand{\code}[1]{\texttt{#1}}
\newcommand{\sensor}[1]{\texttt{#1}}
\newcommand{\finger}[1]{\texttt{#1}}
\newcommand\myindent{\-\hspace{10mm}}
\newcommand\tabindent{\hspace{1em}}
\newcommand\mat[1]{\boldsymbol{#1}}
\newcommand\vect[1]{\boldsymbol{#1}}
\newcommand\mate[1]{\widetilde{\mat{#1}}}
\newcommand\vecte[1]{\widetilde{\vect{#1}}}
\newcommand{\norm}[1]{\left\lVert#1\right\rVert}
\newcommand\asin[1]{\arcsin{\left(#1\right)}}
\newcommand\atan[2]{\arctan{\left(\dfrac{#1}{#2}\right)}}
\newcommand\Sin[1]{s_{#1}}
\newcommand\Cos[1]{c_{#1}}
\DeclareMathOperator{\sign}{sign}
\newcommand\lnM[1]{\ln\left( #1\right) }
\newcommand\vectorP[1]{\innervectorP(#1)}
\def\innervectorP(#1,#2,#3){\left[#1,\,#2,\,#3 \right]}
\newcommand\vectorT[1]{\innervectorT(#1)}
\def\innervectorT(#1,#2,#3,#4,#5,#6){\left[#1,\,#2,\,#3,\,#4,\,#5,\,#6 \right]}
\newcommand{\minus}{\scalebox{0.5}[1.0]{$-$}}
\newcommand{\matrixspace}[0]{\vspace{0.75ex}}
\newcommand{\figspace}[0]{\vspace{7.5mm}}
\newcommand*\circled[1]{\tikz[baseline=(char.base)]{\node[shape=circle,draw,inner sep=1pt] (char) {#1};}}
\newcommand{\bs}{\texttt{x}}
\newcommand{\bt}{\texttt{o}\,}

% Algorithmic modifications
\makeatletter
\renewcommand{\algorithmicrequire}{\textbf{Input:}}
\renewcommand{\algorithmicensure}{\textbf{Output:}}
\newcommand{\algorithmicbreak}{\textbf{break}}
\newcommand{\BREAK}{\STATE \algorithmicbreak}
\makeatother

\hyphenation{%
acce-le-ra-tion
dis-po-si-ti-vos
ge-ne-ra-li-za-tion
me-thods 
sig-ni-fi-can-tly
ta-xo-no-mies
te-le-ma-ni-pu-la-tion}

% Helper lengths
\newlength\figheight
\newlength\figwidth
\setlength\figheight{40mm} % Default one
\newlength\handsize

%% ----------------------------------------------------------------
\begin{document}
\frontmatter    % Begin Roman style (i, ii, iii, iv...) page numbering

% Set up the Title Page
\title    {Unnecessarily Complicated Research Title}
\authors    {Francisco Su\'{a}rez Ruiz}
\authordegree {Ingeniero Mecatr\'{o}nico} 
\supervisor {D. Manuel Ferre P\'{e}rez}
\superdegree{Doctor Ingeniero Industrial}
% Jury
\president    {Dr. Presidente}
\secretary    {Dr. Secretario}
\vocalA       {Dr. Vocal A}
\vocalB       {Dr. Vocal B}
\vocalC       {Dr. Vocal C}
\surrogateA   {Dr. Suplente A}
\surrogateB   {Dr. Suplente B}
% Do not change this here, instead these must be set in the "UPMthesis.cls" file, please look through it instead
\addresses  {\groupname\\\deptname\\\univname}
\date       {2015}
\subject    {}
\keywords   {}

\maketitle
%% ----------------------------------------------------------------
\setstretch{1.3}  % It is better to have smaller font and larger line spacing than the other way round
% Define the page headers using the FancyHdr package and set up for one-sided printing
\fancyhead{}  % Clears all page headers and footers
\rhead{\thepage}  % Sets the right side header to show the page number
\lhead{}  % Clears the left side page header
\pagestyle{fancy}  % Finally, use the "fancy" page style to implement the FancyHdr headers
%% ----------------------------------------------------------------
\setstretch{1.3}
%~ \linenumbers
\begin{acknowledgements}
I firmly believe in the power of collaboration and feel very fortunate about finishing this thesis at this specific time in history. Somehow, it seems to be the time of open source and open access resources. To start with, this whole document has been written using \LaTeX, available for free under the Project Public License. 

\blindtext

This thesis has been an incredible journey and can be summarized quoting a phrase of Bernard of Chartres, popularized by Isaac Newton: ``If I have seen further it is by standing on the shoulders of giants''.

\vfill

This work has been fully funded by the ...
\end{acknowledgements}

\begin{abstract}
\blindtext
\end{abstract}


\begin{resumen}
\blindtext
\end{resumen}


%The page style headers have been "empty" all this time, now use the "fancy" headers as defined before to bring them back
\pagestyle{fancy}
\fancyhead[LE,RO]{\thepage}
\fancyhead[RE]{\textit{\nouppercase{\leftmark}}}
\fancyhead[LO]{\textit{\nouppercase{\rightmark}}}
%% ----------------------------------------------------------------
% Hack for that annoying extra blank page after table of contents
{\begingroup
\let\cleardoublepage\relax
\let\clearpage\relax
\tableofcontents
\endgroup}
\clearpage{\pagestyle{empty}\cleardoublepage}
{\begingroup
\let\cleardoublepage\relax
\let\clearpage\relax
\printglossaries
\endgroup}

\cleardoublepage
%% ----------------------------------------------------------------
% Begin the Dedication page
\setstretch{1.3}  % Return the line spacing back to 1.3
\pagestyle{empty}  % Page style needs to be empty for this page
\dedicatory{To my family...}
\addtocontents{toc}{\vspace{1em}}  % Add a gap in the Contents, for aesthetics
%% ----------------------------------------------------------------
\mainmatter   % Begin normal, numeric (1,2,3...) page numbering
%The page style headers have been "empty" all this time, now use the "fancy" headers as defined before to bring them back
\pagestyle{fancy}
\fancyhead[LE,RO]{\thepage}
\fancyhead[RE]{\textit{\nouppercase{\leftmark}}}
\fancyhead[LO]{\textit{\nouppercase{\rightmark}}}

%% ----------------------------------------------------------------
% Chapters
\chapter{Introduction}
\label{ch:introduction}
\blindtext

\section{Research Motivation}
\blindtext

\section{Contributions}
\blindtext
\blinditemize


\section{Outline}
\blindtext

\chapter{Conclusions}
\label{ch:conclusions}
\blindtext

\section{Future Work}
\blindtext


%% ----------------------------------------------------------------
% Now begin the Appendices, including them as separate files
\addtocontents{toc}{\vspace{2em}} % Add a gap in the Contents, for aesthetics
\appendix

\chapter{Kinematics Using Displacement Matrices}
\label{ch:kinematics}

The method proposed by \citeauthor{Denavit1955} \cite{Denavit1955,Paul1981} is the most popular to solve the \acf{fk} of a manipulator. It assigns a coordinate system to each link at the joint axis and then expresses the relationship between consecutive coordinate systems with homogeneous transformation matrices. All of the individual link transformation matrices may then be multiplied together to produce one transformation that relates the coordinate system at the end-effector to the base coordinate frame. The resulting matrix is a function of the joint displacement variables and the various parameters that describe the geometry of the manipulator.

To simplify the derivation of each link transformation matrix, \citeauthor{Denavit1955} developed a set of rules for locating the various coordinate systems.
Following these rules ensures that every transformation matrix has the same functional form and that the geometric parameters are well defined.

When the manipulator has special kinematic configuration, these rules require extra ``dummy'' links to maintain the consistency between coordinate frames. Many alternatives exist to deal with such special configurations but they have been eclipsed by the popularity of the \ac{dh} method. Some examples are the use of dual numbers \cite{Veldkamp1976,McCarthy1986}, dual quaternions \cite{Perez2004} and product of exponentials \cite{Brockett1984}.

In this work, the displacement matrices method \cite{Barrientos2012} is used. This method leads to the same result as the product of exponentials but avoids the complex selection of the \ac{dh} coordinate frames. 


\section{Forward Kinematics}

The displacement matrices method only requires to identify the axis of movement for each joint (rotation or translation depending on the joint type), a $(x,y,z)$ point over this axis and the transformation $\mat{T}_0$ that relates the coordinate system at the end-effector to the base coordinate frame when the robot is in the initial position ($q_i=0\;\forall{}\;i$)

\begin{algorithm}[htbp]
\caption{Forward kinematics using the displacement matrices method}
\label{alg:fk}
\begin{algorithmic}[1]
  \REQUIRE $n > 0,\, \mat{T}_0,\, \vect{k}_i\;\forall{}\;i ,\, \vect{p}_i\;\forall{}\;i $
  \ENSURE $\mat{T}_n :=$ end-effector transformation \\
  \COMMENT{Vectors and transformations referred to the robot base frame}
  \STATE $n \leftarrow$ number of DoF
  \STATE $\mat{T}_0 \leftarrow$ transformation at home position $\left(q_i=0\;\forall{}\;i\right)$
  \matrixspace
  \STATE $\mat{I} \leftarrow \begin{bmatrix}
    1   & 0   & 0  \\
    0   & 1   & 0  \\
    0   & 0   & 1
    \end{bmatrix}$ 
  \matrixspace
  \STATE $\mat{T}_n \leftarrow \begin{bmatrix}
    1   & 0   & 0  & 0  \\
    0   & 1   & 0  & 0  \\
    0   & 0   & 1  & 0  \\
    0   & 0   & 0  & 1
    \end{bmatrix}$
  \matrixspace
  \FOR{$i = 1$ \TO $n$}
  \STATE $\vect{k}_i \leftarrow$ joint axis direction vector, $\mid\vect{k}_i\mid = 1$
  \STATE $\vect{p}_i \leftarrow$ any point $\left(x,y,z\right)$ over the axis
  \IF{joint $i$ is \textit{prismatic}} \label{line:fk_if_type}
  \matrixspace
  \STATE $\mat{D}_i \leftarrow \begin{bmatrix}
    \mat{I} & \vect{k}_i\,^T\cdot q_i\\
    0       & 1
    \end{bmatrix}$
  \matrixspace
  \ELSIF{joint $i$ is \textit{revolute}}
  \matrixspace
  \STATE $\mbox{skew}\left(\vect{k}\right) \leftarrow \begin{bmatrix}
    0     & -k_z  & k_y  \\
    k_z   & 0     & -k_x  \\
    -k_y  & k_x   & 0
    \end{bmatrix}$
  \matrixspace
  \STATE $\mat{R}_i \leftarrow \mat{I}\cos{q_i}+\vect{k}_i\vect{k}_i^T\left(1-\cos{q_i}\right) + \mbox{skew}\left(\vect{k}_i\right)\,\sin{q_i}$
  \matrixspace
  \STATE $\mat{D}_i \leftarrow \begin{bmatrix}
    \mat{R}_i   & \left(\mat{I} - \mat{R}_i\right)\vect{p}_i^T\\
    0   & 1
    \end{bmatrix}$
  \matrixspace
  \ENDIF \label{line:fk_endif_type}
  \STATE $\mat{T}_n \leftarrow \mat{T}_n\mat{D}_i$
  \ENDFOR
  \STATE $\mat{T}_n \leftarrow \mat{T}_n\mat{T}_0$
  \RETURN $\mat{T}_n$
\end{algorithmic}
\end{algorithm}

\mref{alg:fk} shows that, similar to the \ac{dh} convention \cite{Denavit1955}, the displacement matrices method can be systematically implemented.

Place the robot at its initial position ($q_i=0\;\forall{}\;i$), find the homogeneous transformation $\mat{T}_0$ that locates the robot's end-effector reference frame relative to the base frame.

In the same position and for each \ac{dof} $q_i$ obtain the displacement matrix $\mat{D}_i$:
\begin{itemize}
\item Identify the joint axis direction vector $\vect{k}_i$ relative to the base frame.
\item Select any point $\vect{p}_i$ over the joint axis.
\item Calculate the displacement matrix $\mat{D}_i$ associated to the \ac{dof} depending on the type of joint: prismatic or revolute, see \mref{alg:fk} at lines \ref{line:fk_if_type}-\ref{line:fk_endif_type}.
\item The FK is obtained as: $\mat{T}_n=\left(\prod_{i=0}^n\mat{D}_i\right)\cdot\mat{T}_0$
\end{itemize}

\section{Differential Kinematics}

\begin{algorithm}
\caption{Computation of the Geometric Jacobian using displacement matrices}
\label{alg:jacobian}
\begin{algorithmic}[1]
  \REQUIRE $\mat{T}_n, \;\mat{D}_i\;\forall{}\;i, \;\vect{k}_i\;\forall{}\;i, \;\vect{p}_i\;\forall{}\;i$
  \ENSURE $\mat{J} :=$ Geometric Jacobian\\
  \STATE $n \leftarrow$ robot \acp{dof}

  \STATE $\mat{R}_j \leftarrow \begin{bmatrix}
      1   & 0   & 0  \\
      0   & 1   & 0  \\
      0   & 0   & 1
    \end{bmatrix}$
  \STATE $\mat{D}_j \leftarrow \begin{bmatrix}
      1   & 0   & 0  & 0  \\
      0   & 1   & 0  & 0  \\
      0   & 0   & 1  & 0  \\
      0   & 0   & 0  & 1    
    \end{bmatrix}$
  \STATE $^0\vect{p}_e \leftarrow \mat{T}_n\left[1:3,4\right]$
  \COMMENT{End-effector position $\left(x,y,z\right)$}
  \FOR{$i = 0$ \TO $n$}
  \STATE $\mat{R}_j \leftarrow \mat{R}_j \cdot \mat{D}_i\left[1:3,1:3\right]$
  \COMMENT{Rotation part of the displacement matrix}
  \STATE $\vect{k}_j \leftarrow \mat{R}_j \cdot \vect{k}_i\,^T$
  \STATE $\mat{D}_j \leftarrow \mat{D}_j \cdot \mat{D}_i$
  \IF{joint $i$ is \textit{prismatic}}
  \STATE $\mat{J}\left[:,i\right] \leftarrow \begin{bmatrix} \vect{k}_j \\  0 \end{bmatrix}$
  \COMMENT{Populate the jacobian column by column}
  \ELSIF{joint $i$ is \textit{revolute}}
  \STATE $\begin{bmatrix} \vect{p}_j\\ 1 \end{bmatrix} \leftarrow \mat{D}_j \cdot \begin{bmatrix} \vect{p}_i\,^T\\ 1 \end{bmatrix}$
  \STATE ${^j\vect{p}}_e \leftarrow {^0\vect{p}}_e - \vect{p}_j$
  \STATE $\mat{J}\left[:,i\right] \leftarrow \begin{bmatrix} \vect{k}_j\times{^j\vect{p}}_e \\  \vect{k}_j \end{bmatrix}$
  \COMMENT{Populate the jacobian column by column}
  \ENDIF
  \ENDFOR
  \RETURN $\mat{J}$
\end{algorithmic}
\end{algorithm}

Besides the \ac{fk} information, the differential kinematics is an important tool for robot control an analysis.
\begin{definition}
Differential kinematics is the relationship between motion in the joint space and the motion in the task space.
\end{definition}

This relationship is given by the Jacobian matrix $\mat{J}$, for instance for the velocity case,
\begin{equation}
  \begin{bmatrix} v_x \\ v_y \\ v_z \\ w_x \\ w_y \\ w_z\end{bmatrix} =
  \mat{J}
  \begin{bmatrix} q_1 \\ q_2 \\ \vdots \\ q_n\end{bmatrix}
\end{equation}

Additionally, forces in the task space can be transformed into torques in the joint space,

\begin{equation}
  \vect{\tau} = \mat{J}^T\,\vect{w} \Rightarrow
  \begin{bmatrix} \tau_1 \\ \tau_2 \\ \vdots \\ \tau_n\end{bmatrix} = 
  \mat{J}^T
  \begin{bmatrix} \vect{f} \\ \vect{f}\times\vect{p} \end{bmatrix}
\end{equation}

where $\vect{\tau}$ is the torque command to the joints, $\vect{w}$ is the desired wrench (force and torque assembled screw) and $\mat{J}^T$ is the transpose of the robot Jacobian.

\mref{alg:jacobian} shows how to obtain the columns of the Geometric Jacobian using the displacement matrices parameters.


\addtocontents{toc}{\vspace{1em}}  % Add a gap in the Contents, for aesthetics
\listoffigures
\listoftables

\backmatter

%% ----------------------------------------------------------------
\printbibliography
\end{document}  % The End
%% ----------------------------------------------------------------
